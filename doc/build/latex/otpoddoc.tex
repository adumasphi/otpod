% Generated by Sphinx.
\def\sphinxdocclass{report}
\documentclass[letterpaper,10pt,english]{sphinxmanual}
\usepackage[utf8]{inputenc}
\DeclareUnicodeCharacter{00A0}{\nobreakspace}
\usepackage{cmap}
\usepackage[T1]{fontenc}
\usepackage{babel}
\usepackage{times}
\usepackage[Bjarne]{fncychap}
\usepackage{longtable}
\usepackage{sphinx}
\usepackage{multirow}
\usepackage{eqparbox}


\addto\captionsenglish{\renewcommand{\figurename}{Fig. }}
\addto\captionsenglish{\renewcommand{\tablename}{Table }}
\SetupFloatingEnvironment{literal-block}{name=Listing }



\title{otpod Documentation}
\date{April 19, 2016}
\release{0.0.1}
\author{Antoine Dumas}
\newcommand{\sphinxlogo}{}
\renewcommand{\releasename}{Release}
\setcounter{tocdepth}{2}
\makeindex

\makeatletter
\def\PYG@reset{\let\PYG@it=\relax \let\PYG@bf=\relax%
    \let\PYG@ul=\relax \let\PYG@tc=\relax%
    \let\PYG@bc=\relax \let\PYG@ff=\relax}
\def\PYG@tok#1{\csname PYG@tok@#1\endcsname}
\def\PYG@toks#1+{\ifx\relax#1\empty\else%
    \PYG@tok{#1}\expandafter\PYG@toks\fi}
\def\PYG@do#1{\PYG@bc{\PYG@tc{\PYG@ul{%
    \PYG@it{\PYG@bf{\PYG@ff{#1}}}}}}}
\def\PYG#1#2{\PYG@reset\PYG@toks#1+\relax+\PYG@do{#2}}

\expandafter\def\csname PYG@tok@gd\endcsname{\def\PYG@tc##1{\textcolor[rgb]{0.63,0.00,0.00}{##1}}}
\expandafter\def\csname PYG@tok@gu\endcsname{\let\PYG@bf=\textbf\def\PYG@tc##1{\textcolor[rgb]{0.50,0.00,0.50}{##1}}}
\expandafter\def\csname PYG@tok@gt\endcsname{\def\PYG@tc##1{\textcolor[rgb]{0.00,0.27,0.87}{##1}}}
\expandafter\def\csname PYG@tok@gs\endcsname{\let\PYG@bf=\textbf}
\expandafter\def\csname PYG@tok@gr\endcsname{\def\PYG@tc##1{\textcolor[rgb]{1.00,0.00,0.00}{##1}}}
\expandafter\def\csname PYG@tok@cm\endcsname{\let\PYG@it=\textit\def\PYG@tc##1{\textcolor[rgb]{0.38,0.63,0.69}{##1}}}
\expandafter\def\csname PYG@tok@vg\endcsname{\def\PYG@tc##1{\textcolor[rgb]{0.73,0.38,0.84}{##1}}}
\expandafter\def\csname PYG@tok@vi\endcsname{\def\PYG@tc##1{\textcolor[rgb]{0.73,0.38,0.84}{##1}}}
\expandafter\def\csname PYG@tok@mh\endcsname{\def\PYG@tc##1{\textcolor[rgb]{0.25,0.63,0.44}{##1}}}
\expandafter\def\csname PYG@tok@cs\endcsname{\def\PYG@tc##1{\textcolor[rgb]{0.38,0.63,0.69}{##1}}\def\PYG@bc##1{\setlength{\fboxsep}{0pt}\colorbox[rgb]{1.00,0.94,0.94}{\strut ##1}}}
\expandafter\def\csname PYG@tok@ge\endcsname{\let\PYG@it=\textit}
\expandafter\def\csname PYG@tok@vc\endcsname{\def\PYG@tc##1{\textcolor[rgb]{0.73,0.38,0.84}{##1}}}
\expandafter\def\csname PYG@tok@il\endcsname{\def\PYG@tc##1{\textcolor[rgb]{0.25,0.63,0.44}{##1}}}
\expandafter\def\csname PYG@tok@go\endcsname{\def\PYG@tc##1{\textcolor[rgb]{0.53,0.53,0.53}{##1}}}
\expandafter\def\csname PYG@tok@cp\endcsname{\def\PYG@tc##1{\textcolor[rgb]{0.00,0.44,0.13}{##1}}}
\expandafter\def\csname PYG@tok@gi\endcsname{\def\PYG@tc##1{\textcolor[rgb]{0.00,0.63,0.00}{##1}}}
\expandafter\def\csname PYG@tok@gh\endcsname{\let\PYG@bf=\textbf\def\PYG@tc##1{\textcolor[rgb]{0.00,0.00,0.50}{##1}}}
\expandafter\def\csname PYG@tok@ni\endcsname{\let\PYG@bf=\textbf\def\PYG@tc##1{\textcolor[rgb]{0.84,0.33,0.22}{##1}}}
\expandafter\def\csname PYG@tok@nl\endcsname{\let\PYG@bf=\textbf\def\PYG@tc##1{\textcolor[rgb]{0.00,0.13,0.44}{##1}}}
\expandafter\def\csname PYG@tok@nn\endcsname{\let\PYG@bf=\textbf\def\PYG@tc##1{\textcolor[rgb]{0.05,0.52,0.71}{##1}}}
\expandafter\def\csname PYG@tok@no\endcsname{\def\PYG@tc##1{\textcolor[rgb]{0.38,0.68,0.84}{##1}}}
\expandafter\def\csname PYG@tok@na\endcsname{\def\PYG@tc##1{\textcolor[rgb]{0.25,0.44,0.63}{##1}}}
\expandafter\def\csname PYG@tok@nb\endcsname{\def\PYG@tc##1{\textcolor[rgb]{0.00,0.44,0.13}{##1}}}
\expandafter\def\csname PYG@tok@nc\endcsname{\let\PYG@bf=\textbf\def\PYG@tc##1{\textcolor[rgb]{0.05,0.52,0.71}{##1}}}
\expandafter\def\csname PYG@tok@nd\endcsname{\let\PYG@bf=\textbf\def\PYG@tc##1{\textcolor[rgb]{0.33,0.33,0.33}{##1}}}
\expandafter\def\csname PYG@tok@ne\endcsname{\def\PYG@tc##1{\textcolor[rgb]{0.00,0.44,0.13}{##1}}}
\expandafter\def\csname PYG@tok@nf\endcsname{\def\PYG@tc##1{\textcolor[rgb]{0.02,0.16,0.49}{##1}}}
\expandafter\def\csname PYG@tok@si\endcsname{\let\PYG@it=\textit\def\PYG@tc##1{\textcolor[rgb]{0.44,0.63,0.82}{##1}}}
\expandafter\def\csname PYG@tok@s2\endcsname{\def\PYG@tc##1{\textcolor[rgb]{0.25,0.44,0.63}{##1}}}
\expandafter\def\csname PYG@tok@nt\endcsname{\let\PYG@bf=\textbf\def\PYG@tc##1{\textcolor[rgb]{0.02,0.16,0.45}{##1}}}
\expandafter\def\csname PYG@tok@nv\endcsname{\def\PYG@tc##1{\textcolor[rgb]{0.73,0.38,0.84}{##1}}}
\expandafter\def\csname PYG@tok@s1\endcsname{\def\PYG@tc##1{\textcolor[rgb]{0.25,0.44,0.63}{##1}}}
\expandafter\def\csname PYG@tok@ch\endcsname{\let\PYG@it=\textit\def\PYG@tc##1{\textcolor[rgb]{0.38,0.63,0.69}{##1}}}
\expandafter\def\csname PYG@tok@m\endcsname{\def\PYG@tc##1{\textcolor[rgb]{0.25,0.63,0.44}{##1}}}
\expandafter\def\csname PYG@tok@gp\endcsname{\let\PYG@bf=\textbf\def\PYG@tc##1{\textcolor[rgb]{0.78,0.36,0.04}{##1}}}
\expandafter\def\csname PYG@tok@sh\endcsname{\def\PYG@tc##1{\textcolor[rgb]{0.25,0.44,0.63}{##1}}}
\expandafter\def\csname PYG@tok@ow\endcsname{\let\PYG@bf=\textbf\def\PYG@tc##1{\textcolor[rgb]{0.00,0.44,0.13}{##1}}}
\expandafter\def\csname PYG@tok@sx\endcsname{\def\PYG@tc##1{\textcolor[rgb]{0.78,0.36,0.04}{##1}}}
\expandafter\def\csname PYG@tok@bp\endcsname{\def\PYG@tc##1{\textcolor[rgb]{0.00,0.44,0.13}{##1}}}
\expandafter\def\csname PYG@tok@c1\endcsname{\let\PYG@it=\textit\def\PYG@tc##1{\textcolor[rgb]{0.38,0.63,0.69}{##1}}}
\expandafter\def\csname PYG@tok@o\endcsname{\def\PYG@tc##1{\textcolor[rgb]{0.40,0.40,0.40}{##1}}}
\expandafter\def\csname PYG@tok@kc\endcsname{\let\PYG@bf=\textbf\def\PYG@tc##1{\textcolor[rgb]{0.00,0.44,0.13}{##1}}}
\expandafter\def\csname PYG@tok@c\endcsname{\let\PYG@it=\textit\def\PYG@tc##1{\textcolor[rgb]{0.38,0.63,0.69}{##1}}}
\expandafter\def\csname PYG@tok@mf\endcsname{\def\PYG@tc##1{\textcolor[rgb]{0.25,0.63,0.44}{##1}}}
\expandafter\def\csname PYG@tok@err\endcsname{\def\PYG@bc##1{\setlength{\fboxsep}{0pt}\fcolorbox[rgb]{1.00,0.00,0.00}{1,1,1}{\strut ##1}}}
\expandafter\def\csname PYG@tok@mb\endcsname{\def\PYG@tc##1{\textcolor[rgb]{0.25,0.63,0.44}{##1}}}
\expandafter\def\csname PYG@tok@ss\endcsname{\def\PYG@tc##1{\textcolor[rgb]{0.32,0.47,0.09}{##1}}}
\expandafter\def\csname PYG@tok@sr\endcsname{\def\PYG@tc##1{\textcolor[rgb]{0.14,0.33,0.53}{##1}}}
\expandafter\def\csname PYG@tok@mo\endcsname{\def\PYG@tc##1{\textcolor[rgb]{0.25,0.63,0.44}{##1}}}
\expandafter\def\csname PYG@tok@kd\endcsname{\let\PYG@bf=\textbf\def\PYG@tc##1{\textcolor[rgb]{0.00,0.44,0.13}{##1}}}
\expandafter\def\csname PYG@tok@mi\endcsname{\def\PYG@tc##1{\textcolor[rgb]{0.25,0.63,0.44}{##1}}}
\expandafter\def\csname PYG@tok@kn\endcsname{\let\PYG@bf=\textbf\def\PYG@tc##1{\textcolor[rgb]{0.00,0.44,0.13}{##1}}}
\expandafter\def\csname PYG@tok@cpf\endcsname{\let\PYG@it=\textit\def\PYG@tc##1{\textcolor[rgb]{0.38,0.63,0.69}{##1}}}
\expandafter\def\csname PYG@tok@kr\endcsname{\let\PYG@bf=\textbf\def\PYG@tc##1{\textcolor[rgb]{0.00,0.44,0.13}{##1}}}
\expandafter\def\csname PYG@tok@s\endcsname{\def\PYG@tc##1{\textcolor[rgb]{0.25,0.44,0.63}{##1}}}
\expandafter\def\csname PYG@tok@kp\endcsname{\def\PYG@tc##1{\textcolor[rgb]{0.00,0.44,0.13}{##1}}}
\expandafter\def\csname PYG@tok@w\endcsname{\def\PYG@tc##1{\textcolor[rgb]{0.73,0.73,0.73}{##1}}}
\expandafter\def\csname PYG@tok@kt\endcsname{\def\PYG@tc##1{\textcolor[rgb]{0.56,0.13,0.00}{##1}}}
\expandafter\def\csname PYG@tok@sc\endcsname{\def\PYG@tc##1{\textcolor[rgb]{0.25,0.44,0.63}{##1}}}
\expandafter\def\csname PYG@tok@sb\endcsname{\def\PYG@tc##1{\textcolor[rgb]{0.25,0.44,0.63}{##1}}}
\expandafter\def\csname PYG@tok@k\endcsname{\let\PYG@bf=\textbf\def\PYG@tc##1{\textcolor[rgb]{0.00,0.44,0.13}{##1}}}
\expandafter\def\csname PYG@tok@se\endcsname{\let\PYG@bf=\textbf\def\PYG@tc##1{\textcolor[rgb]{0.25,0.44,0.63}{##1}}}
\expandafter\def\csname PYG@tok@sd\endcsname{\let\PYG@it=\textit\def\PYG@tc##1{\textcolor[rgb]{0.25,0.44,0.63}{##1}}}

\def\PYGZbs{\char`\\}
\def\PYGZus{\char`\_}
\def\PYGZob{\char`\{}
\def\PYGZcb{\char`\}}
\def\PYGZca{\char`\^}
\def\PYGZam{\char`\&}
\def\PYGZlt{\char`\<}
\def\PYGZgt{\char`\>}
\def\PYGZsh{\char`\#}
\def\PYGZpc{\char`\%}
\def\PYGZdl{\char`\$}
\def\PYGZhy{\char`\-}
\def\PYGZsq{\char`\'}
\def\PYGZdq{\char`\"}
\def\PYGZti{\char`\~}
% for compatibility with earlier versions
\def\PYGZat{@}
\def\PYGZlb{[}
\def\PYGZrb{]}
\makeatother

\renewcommand\PYGZsq{\textquotesingle}

\begin{document}

\maketitle
\tableofcontents
\phantomsection\label{index::doc}



\chapter{Contents:}
\label{index:welcome-to-otpod-s-documentation}\label{index:contents}

\section{Documentation of the API}
\label{user_manual::doc}\label{user_manual:documentation-of-the-api}
This is the user manual for the Python bindings to the otpod library.


\subsection{Data analysis}
\label{user_manual:data-analysis}
\begin{longtable}{ll}
\hline
\endfirsthead

\multicolumn{2}{c}%
{{\textsf{\tablename\ \thetable{} -- continued from previous page}}} \\
\hline
\endhead

\hline \multicolumn{2}{|r|}{{\textsf{Continued on next page}}} \\ \hline
\endfoot

\endlastfoot


{\hyperref[_generated/otpod.UnivariateLinearModelAnalysis:otpod.UnivariateLinearModelAnalysis]{\emph{\code{UnivariateLinearModelAnalysis}}}}
 & 
Linear regression analysis with residuals hypothesis tests.
\\
\hline\end{longtable}



\subsubsection{UnivariateLinearModelAnalysis}
\label{_generated/otpod.UnivariateLinearModelAnalysis:univariatelinearmodelanalysis}\label{_generated/otpod.UnivariateLinearModelAnalysis::doc}\index{UnivariateLinearModelAnalysis (class in otpod)}

\begin{fulllineitems}
\phantomsection\label{_generated/otpod.UnivariateLinearModelAnalysis:otpod.UnivariateLinearModelAnalysis}\pysiglinewithargsret{\strong{class }\bfcode{UnivariateLinearModelAnalysis}}{\emph{*args}}{}
Linear regression analysis with residuals hypothesis tests.

\textbf{Available constructors:}

UnivariateLinearModelAnalysis(\emph{inputSample, outputSample})

UnivariateLinearModelAnalysis(\emph{inputSample, outputSample, noiseThres,
saturationThres, resDistFact, boxCox})
\begin{quote}\begin{description}
\item[{Parameters}] \leavevmode
\textbf{inputSample} : 2-d sequence of float
\begin{quote}

Vector of the defect sizes, of dimension 1.
\end{quote}

\textbf{outputSample} : 2-d sequence of float
\begin{quote}

Vector of the signals, of dimension 1.
\end{quote}

\textbf{noiseThres} : float
\begin{quote}

Value for low censored data. Default is None.
\end{quote}

\textbf{saturationThres} : float
\begin{quote}

Value for high censored data. Default is None
\end{quote}

\textbf{resDistFact} : \href{http://doc.openturns.org/openturns-latest/sphinx/user\_manual/\_generated/openturns.DistributionFactory.html\#openturns.DistributionFactory}{\code{openturns.DistributionFactory}}
\begin{quote}

Distribution hypothesis followed by the residuals. Default is 
\href{http://doc.openturns.org/openturns-latest/sphinx/user\_manual/\_generated/openturns.NormalFactory.html\#openturns.NormalFactory}{\code{openturns.NormalFactory}}.
\end{quote}

\textbf{boxCox} : bool or float
\begin{quote}

Enable or not the Box Cox transformation. If boxCox is a float, the Box
Cox transformation is enabled with the given value. Default is False.
\end{quote}

\end{description}\end{quote}
\paragraph{Notes}

This method automatically :
\begin{itemize}
\item {} 
computes the Box Cox parameter if \emph{boxCox} is True,

\item {} 
computes the transformed signals if \emph{boxCox} is True or a float,

\item {} 
builds the univariate linear regression model on the data,

\item {} 
computes the linear regression parameters for censored data if needed,

\item {} 
computes the residuals,

\item {} 
runs all hypothesis tests.

\end{itemize}
\paragraph{Methods}

\begin{longtable}{ll}
\hline
\endfirsthead

\multicolumn{2}{c}%
{{\textsf{\tablename\ \thetable{} -- continued from previous page}}} \\
\hline
\endhead

\hline \multicolumn{2}{|r|}{{\textsf{Continued on next page}}} \\ \hline
\endfoot

\endlastfoot


{\hyperref[_generated/otpod.UnivariateLinearModelAnalysis:otpod.UnivariateLinearModelAnalysis.drawBoxCoxLikelihood]{\emph{\code{drawBoxCoxLikelihood}}}}({[}name{]})
 & 
Draw the loglikelihood versus the Box Cox parameter.
\\
\hline
{\hyperref[_generated/otpod.UnivariateLinearModelAnalysis:otpod.UnivariateLinearModelAnalysis.drawLinearModel]{\emph{\code{drawLinearModel}}}}({[}model, name{]})
 & 
Draw the linear regression prediction versus the true data.
\\
\hline
{\hyperref[_generated/otpod.UnivariateLinearModelAnalysis:otpod.UnivariateLinearModelAnalysis.drawResiduals]{\emph{\code{drawResiduals}}}}({[}model, name{]})
 & 
Draw the residuals versus the defect values.
\\
\hline
{\hyperref[_generated/otpod.UnivariateLinearModelAnalysis:otpod.UnivariateLinearModelAnalysis.drawResidualsDistribution]{\emph{\code{drawResidualsDistribution}}}}({[}model, name{]})
 & 
Draw the residuals histogram with the fitted distribution.
\\
\hline
{\hyperref[_generated/otpod.UnivariateLinearModelAnalysis:otpod.UnivariateLinearModelAnalysis.drawResidualsQQplot]{\emph{\code{drawResidualsQQplot}}}}({[}model, name{]})
 & 
Draw the residuals QQ plot with the fitted distribution.
\\
\hline
{\hyperref[_generated/otpod.UnivariateLinearModelAnalysis:otpod.UnivariateLinearModelAnalysis.getAndersonDarlingPValue]{\emph{\code{getAndersonDarlingPValue}}}}()
 & 
Accessor to the Anderson Darling test p-value.
\\
\hline
{\hyperref[_generated/otpod.UnivariateLinearModelAnalysis:otpod.UnivariateLinearModelAnalysis.getBoxCoxParameter]{\emph{\code{getBoxCoxParameter}}}}()
 & 
Accessor to the Box Cox parameter.
\\
\hline
{\hyperref[_generated/otpod.UnivariateLinearModelAnalysis:otpod.UnivariateLinearModelAnalysis.getBreuschPaganPValue]{\emph{\code{getBreuschPaganPValue}}}}()
 & 
Accessor to the Breusch Pagan test p-value.
\\
\hline
{\hyperref[_generated/otpod.UnivariateLinearModelAnalysis:otpod.UnivariateLinearModelAnalysis.getCramerVonMisesPValue]{\emph{\code{getCramerVonMisesPValue}}}}()
 & 
Accessor to the Cramer Von Mises test p-value.
\\
\hline
{\hyperref[_generated/otpod.UnivariateLinearModelAnalysis:otpod.UnivariateLinearModelAnalysis.getDurbinWatsonPValue]{\emph{\code{getDurbinWatsonPValue}}}}()
 & 
Accessor to the Durbin Watson test p-value.
\\
\hline
{\hyperref[_generated/otpod.UnivariateLinearModelAnalysis:otpod.UnivariateLinearModelAnalysis.getHarrisonMcCabePValue]{\emph{\code{getHarrisonMcCabePValue}}}}()
 & 
Accessor to the Harrison McCabe test p-value.
\\
\hline
{\hyperref[_generated/otpod.UnivariateLinearModelAnalysis:otpod.UnivariateLinearModelAnalysis.getInputSample]{\emph{\code{getInputSample}}}}()
 & 
Accessor to the input sample.
\\
\hline
{\hyperref[_generated/otpod.UnivariateLinearModelAnalysis:otpod.UnivariateLinearModelAnalysis.getIntercept]{\emph{\code{getIntercept}}}}()
 & 
Accessor to the intercept of the linear regression model.
\\
\hline
{\hyperref[_generated/otpod.UnivariateLinearModelAnalysis:otpod.UnivariateLinearModelAnalysis.getKolmogorovPValue]{\emph{\code{getKolmogorovPValue}}}}()
 & 
Accessor to the Kolmogorov test p-value.
\\
\hline
{\hyperref[_generated/otpod.UnivariateLinearModelAnalysis:otpod.UnivariateLinearModelAnalysis.getNoiseThreshold]{\emph{\code{getNoiseThreshold}}}}()
 & 
Accessor to the noise threshold.
\\
\hline
{\hyperref[_generated/otpod.UnivariateLinearModelAnalysis:otpod.UnivariateLinearModelAnalysis.getOutputSample]{\emph{\code{getOutputSample}}}}()
 & 
Accessor to the output sample.
\\
\hline
{\hyperref[_generated/otpod.UnivariateLinearModelAnalysis:otpod.UnivariateLinearModelAnalysis.getR2]{\emph{\code{getR2}}}}()
 & 
Accessor to the R2 value.
\\
\hline
{\hyperref[_generated/otpod.UnivariateLinearModelAnalysis:otpod.UnivariateLinearModelAnalysis.getResiduals]{\emph{\code{getResiduals}}}}()
 & 
Accessor to the residuals.
\\
\hline
{\hyperref[_generated/otpod.UnivariateLinearModelAnalysis:otpod.UnivariateLinearModelAnalysis.getResidualsDistribution]{\emph{\code{getResidualsDistribution}}}}()
 & 
Accessor to the residuals distribution.
\\
\hline
{\hyperref[_generated/otpod.UnivariateLinearModelAnalysis:otpod.UnivariateLinearModelAnalysis.getSaturationThreshold]{\emph{\code{getSaturationThreshold}}}}()
 & 
Accessor to the saturation threshold.
\\
\hline
{\hyperref[_generated/otpod.UnivariateLinearModelAnalysis:otpod.UnivariateLinearModelAnalysis.getSlope]{\emph{\code{getSlope}}}}()
 & 
Accessor to the slope of the linear regression model.
\\
\hline
{\hyperref[_generated/otpod.UnivariateLinearModelAnalysis:otpod.UnivariateLinearModelAnalysis.getStandardError]{\emph{\code{getStandardError}}}}()
 & 
Accessor to the standard error of the estimate.
\\
\hline
{\hyperref[_generated/otpod.UnivariateLinearModelAnalysis:otpod.UnivariateLinearModelAnalysis.getZeroMeanPValue]{\emph{\code{getZeroMeanPValue}}}}()
 & 
Accessor to the Zero Mean test p-value.
\\
\hline
{\hyperref[_generated/otpod.UnivariateLinearModelAnalysis:otpod.UnivariateLinearModelAnalysis.printResults]{\emph{\code{printResults}}}}()
 & 
Print results of the linear analysis in the terminal.
\\
\hline
{\hyperref[_generated/otpod.UnivariateLinearModelAnalysis:otpod.UnivariateLinearModelAnalysis.saveResults]{\emph{\code{saveResults}}}}(name)
 & 
Save all analysis test results in a file.
\\
\hline\end{longtable}

\index{drawBoxCoxLikelihood() (UnivariateLinearModelAnalysis method)}

\begin{fulllineitems}
\phantomsection\label{_generated/otpod.UnivariateLinearModelAnalysis:otpod.UnivariateLinearModelAnalysis.drawBoxCoxLikelihood}\pysiglinewithargsret{\bfcode{drawBoxCoxLikelihood}}{\emph{name=None}}{}
Draw the loglikelihood versus the Box Cox parameter.
\begin{quote}\begin{description}
\item[{Parameters}] \leavevmode
\textbf{name} : string
\begin{quote}

name of the figure to be saved with \emph{transparent} option sets to True
and \emph{bbox\_inches='tight'}. It can be only the file name or the 
full path name. Default is None.
\end{quote}

\item[{Returns}] \leavevmode
\textbf{fig} : \href{http://matplotlib.org/api/figure\_api.html}{matplotlib.figure}
\begin{quote}

Matplotlib figure object.
\end{quote}

\textbf{ax} : \href{http://matplotlib.org/api/axes\_api.html}{matplotlib.axes}
\begin{quote}

Matplotlib axes object.
\end{quote}

\end{description}\end{quote}
\paragraph{Notes}

This method is available only when the parameter \emph{boxCox} is set to True.

\end{fulllineitems}

\index{drawLinearModel() (UnivariateLinearModelAnalysis method)}

\begin{fulllineitems}
\phantomsection\label{_generated/otpod.UnivariateLinearModelAnalysis:otpod.UnivariateLinearModelAnalysis.drawLinearModel}\pysiglinewithargsret{\bfcode{drawLinearModel}}{\emph{model='uncensored'}, \emph{name=None}}{}
Draw the linear regression prediction versus the true data.
\begin{quote}\begin{description}
\item[{Parameters}] \leavevmode
\textbf{model} : string
\begin{quote}

The linear regression model to be used, either \emph{uncensored} or
\emph{censored} if censored threshold were given. Default is \emph{uncensored}.
\end{quote}

\textbf{name} : string
\begin{quote}

name of the figure to be saved with \emph{transparent} option sets to True
and \emph{bbox\_inches='tight'}. It can be only the file name or the 
full path name. Default is None.
\end{quote}

\item[{Returns}] \leavevmode
\textbf{fig} : \href{http://matplotlib.org/api/figure\_api.html}{matplotlib.figure}
\begin{quote}

Matplotlib figure object.
\end{quote}

\textbf{ax} : \href{http://matplotlib.org/api/axes\_api.html}{matplotlib.axes}
\begin{quote}

Matplotlib axes object.
\end{quote}

\end{description}\end{quote}

\end{fulllineitems}

\index{drawResiduals() (UnivariateLinearModelAnalysis method)}

\begin{fulllineitems}
\phantomsection\label{_generated/otpod.UnivariateLinearModelAnalysis:otpod.UnivariateLinearModelAnalysis.drawResiduals}\pysiglinewithargsret{\bfcode{drawResiduals}}{\emph{model='uncensored'}, \emph{name=None}}{}
Draw the residuals versus the defect values.
\begin{quote}\begin{description}
\item[{Parameters}] \leavevmode
\textbf{model} : string
\begin{quote}

The residuals to be used, either \emph{uncensored} or
\emph{censored} if censored threshold were given. Default is \emph{uncensored}.
\end{quote}

\textbf{name} : string
\begin{quote}

name of the figure to be saved with \emph{transparent} option sets to True
and \emph{bbox\_inches='tight'}. It can be only the file name or the 
full path name. Default is None.
\end{quote}

\item[{Returns}] \leavevmode
\textbf{fig} : \href{http://matplotlib.org/api/figure\_api.html}{matplotlib.figure}
\begin{quote}

Matplotlib figure object.
\end{quote}

\textbf{ax} : \href{http://matplotlib.org/api/axes\_api.html}{matplotlib.axes}
\begin{quote}

Matplotlib axes object.
\end{quote}

\end{description}\end{quote}

\end{fulllineitems}

\index{drawResidualsDistribution() (UnivariateLinearModelAnalysis method)}

\begin{fulllineitems}
\phantomsection\label{_generated/otpod.UnivariateLinearModelAnalysis:otpod.UnivariateLinearModelAnalysis.drawResidualsDistribution}\pysiglinewithargsret{\bfcode{drawResidualsDistribution}}{\emph{model='uncensored'}, \emph{name=None}}{}
Draw the residuals histogram with the fitted distribution.
\begin{quote}\begin{description}
\item[{Parameters}] \leavevmode
\textbf{model} : string
\begin{quote}

The residuals to be used, either \emph{uncensored} or
\emph{censored} if censored threshold were given. Default is \emph{uncensored}.
\end{quote}

\textbf{name} : string
\begin{quote}

name of the figure to be saved with \emph{transparent} option sets to True
and \emph{bbox\_inches='tight'}. It can be only the file name or the 
full path name. Default is None.
\end{quote}

\item[{Returns}] \leavevmode
\textbf{fig} : \href{http://matplotlib.org/api/figure\_api.html}{matplotlib.figure}
\begin{quote}

Matplotlib figure object.
\end{quote}

\textbf{ax} : \href{http://matplotlib.org/api/axes\_api.html}{matplotlib.axes}
\begin{quote}

Matplotlib axes object.
\end{quote}

\end{description}\end{quote}

\end{fulllineitems}

\index{drawResidualsQQplot() (UnivariateLinearModelAnalysis method)}

\begin{fulllineitems}
\phantomsection\label{_generated/otpod.UnivariateLinearModelAnalysis:otpod.UnivariateLinearModelAnalysis.drawResidualsQQplot}\pysiglinewithargsret{\bfcode{drawResidualsQQplot}}{\emph{model='uncensored'}, \emph{name=None}}{}
Draw the residuals QQ plot with the fitted distribution.
\begin{quote}\begin{description}
\item[{Parameters}] \leavevmode
\textbf{model} : string
\begin{quote}

The residuals to be used, either \emph{uncensored} or
\emph{censored} if censored threshold were given. Default is \emph{uncensored}.
\end{quote}

\textbf{name} : string
\begin{quote}

name of the figure to be saved with \emph{transparent} option sets to True
and \emph{bbox\_inches='tight'}. It can be only the file name or the 
full path name. Default is None.
\end{quote}

\item[{Returns}] \leavevmode
\textbf{fig} : \href{http://matplotlib.org/api/figure\_api.html}{matplotlib.figure}
\begin{quote}

Matplotlib figure object.
\end{quote}

\textbf{ax} : \href{http://matplotlib.org/api/axes\_api.html}{matplotlib.axes}
\begin{quote}

Matplotlib axes object.
\end{quote}

\end{description}\end{quote}

\end{fulllineitems}

\index{getAndersonDarlingPValue() (UnivariateLinearModelAnalysis method)}

\begin{fulllineitems}
\phantomsection\label{_generated/otpod.UnivariateLinearModelAnalysis:otpod.UnivariateLinearModelAnalysis.getAndersonDarlingPValue}\pysiglinewithargsret{\bfcode{getAndersonDarlingPValue}}{}{}
Accessor to the Anderson Darling test p-value.
\begin{quote}\begin{description}
\item[{Returns}] \leavevmode
\textbf{pValue} : \href{http://doc.openturns.org/openturns-latest/sphinx/user\_manual/\_generated/openturns.NumericalPoint.html\#openturns.NumericalPoint}{\code{openturns.NumericalPoint}}
\begin{quote}

Either the p-value for the uncensored case or for both cases.
\end{quote}

\end{description}\end{quote}

\end{fulllineitems}

\index{getBoxCoxParameter() (UnivariateLinearModelAnalysis method)}

\begin{fulllineitems}
\phantomsection\label{_generated/otpod.UnivariateLinearModelAnalysis:otpod.UnivariateLinearModelAnalysis.getBoxCoxParameter}\pysiglinewithargsret{\bfcode{getBoxCoxParameter}}{}{}
Accessor to the Box Cox parameter.
\begin{quote}\begin{description}
\item[{Returns}] \leavevmode
\textbf{lambdaBoxCox} : float
\begin{quote}

The Box Cox parameter used to transform the data. If the transformation
is not enabled None is returned.
\end{quote}

\end{description}\end{quote}

\end{fulllineitems}

\index{getBreuschPaganPValue() (UnivariateLinearModelAnalysis method)}

\begin{fulllineitems}
\phantomsection\label{_generated/otpod.UnivariateLinearModelAnalysis:otpod.UnivariateLinearModelAnalysis.getBreuschPaganPValue}\pysiglinewithargsret{\bfcode{getBreuschPaganPValue}}{}{}
Accessor to the Breusch Pagan test p-value.
\begin{quote}\begin{description}
\item[{Returns}] \leavevmode
\textbf{pValue} : \href{http://doc.openturns.org/openturns-latest/sphinx/user\_manual/\_generated/openturns.NumericalPoint.html\#openturns.NumericalPoint}{\code{openturns.NumericalPoint}}
\begin{quote}

Either the p-value for the uncensored case or for both cases.
\end{quote}

\end{description}\end{quote}

\end{fulllineitems}

\index{getCramerVonMisesPValue() (UnivariateLinearModelAnalysis method)}

\begin{fulllineitems}
\phantomsection\label{_generated/otpod.UnivariateLinearModelAnalysis:otpod.UnivariateLinearModelAnalysis.getCramerVonMisesPValue}\pysiglinewithargsret{\bfcode{getCramerVonMisesPValue}}{}{}
Accessor to the Cramer Von Mises test p-value.
\begin{quote}\begin{description}
\item[{Returns}] \leavevmode
\textbf{pValue} : \href{http://doc.openturns.org/openturns-latest/sphinx/user\_manual/\_generated/openturns.NumericalPoint.html\#openturns.NumericalPoint}{\code{openturns.NumericalPoint}}
\begin{quote}

Either the p-value for the uncensored case or for both cases.
\end{quote}

\end{description}\end{quote}

\end{fulllineitems}

\index{getDurbinWatsonPValue() (UnivariateLinearModelAnalysis method)}

\begin{fulllineitems}
\phantomsection\label{_generated/otpod.UnivariateLinearModelAnalysis:otpod.UnivariateLinearModelAnalysis.getDurbinWatsonPValue}\pysiglinewithargsret{\bfcode{getDurbinWatsonPValue}}{}{}
Accessor to the Durbin Watson test p-value.
\begin{quote}\begin{description}
\item[{Returns}] \leavevmode
\textbf{pValue} : \href{http://doc.openturns.org/openturns-latest/sphinx/user\_manual/\_generated/openturns.NumericalPoint.html\#openturns.NumericalPoint}{\code{openturns.NumericalPoint}}
\begin{quote}

Either the p-value for the uncensored case or for both cases.
\end{quote}

\end{description}\end{quote}

\end{fulllineitems}

\index{getHarrisonMcCabePValue() (UnivariateLinearModelAnalysis method)}

\begin{fulllineitems}
\phantomsection\label{_generated/otpod.UnivariateLinearModelAnalysis:otpod.UnivariateLinearModelAnalysis.getHarrisonMcCabePValue}\pysiglinewithargsret{\bfcode{getHarrisonMcCabePValue}}{}{}
Accessor to the Harrison McCabe test p-value.
\begin{quote}\begin{description}
\item[{Returns}] \leavevmode
\textbf{pValue} : \href{http://doc.openturns.org/openturns-latest/sphinx/user\_manual/\_generated/openturns.NumericalPoint.html\#openturns.NumericalPoint}{\code{openturns.NumericalPoint}}
\begin{quote}

Either the p-value for the uncensored case or for both cases.
\end{quote}

\end{description}\end{quote}

\end{fulllineitems}

\index{getInputSample() (UnivariateLinearModelAnalysis method)}

\begin{fulllineitems}
\phantomsection\label{_generated/otpod.UnivariateLinearModelAnalysis:otpod.UnivariateLinearModelAnalysis.getInputSample}\pysiglinewithargsret{\bfcode{getInputSample}}{}{}
Accessor to the input sample.
\begin{quote}\begin{description}
\item[{Returns}] \leavevmode
\textbf{defects} : \href{http://doc.openturns.org/openturns-latest/sphinx/user\_manual/\_generated/openturns.NumericalSample.html\#openturns.NumericalSample}{\code{openturns.NumericalSample}}
\begin{quote}

The input sample which is the defect values.
\end{quote}

\end{description}\end{quote}

\end{fulllineitems}

\index{getIntercept() (UnivariateLinearModelAnalysis method)}

\begin{fulllineitems}
\phantomsection\label{_generated/otpod.UnivariateLinearModelAnalysis:otpod.UnivariateLinearModelAnalysis.getIntercept}\pysiglinewithargsret{\bfcode{getIntercept}}{}{}
Accessor to the intercept of the linear regression model.
\begin{quote}\begin{description}
\item[{Returns}] \leavevmode
\textbf{intercept} : \href{http://doc.openturns.org/openturns-latest/sphinx/user\_manual/\_generated/openturns.NumericalPoint.html\#openturns.NumericalPoint}{\code{openturns.NumericalPoint}}
\begin{quote}

The intercept parameter for the uncensored and censored (if so) linear
regression model.
\end{quote}

\end{description}\end{quote}

\end{fulllineitems}

\index{getKolmogorovPValue() (UnivariateLinearModelAnalysis method)}

\begin{fulllineitems}
\phantomsection\label{_generated/otpod.UnivariateLinearModelAnalysis:otpod.UnivariateLinearModelAnalysis.getKolmogorovPValue}\pysiglinewithargsret{\bfcode{getKolmogorovPValue}}{}{}
Accessor to the Kolmogorov test p-value.
\begin{quote}\begin{description}
\item[{Returns}] \leavevmode
\textbf{pValue} : \href{http://doc.openturns.org/openturns-latest/sphinx/user\_manual/\_generated/openturns.NumericalPoint.html\#openturns.NumericalPoint}{\code{openturns.NumericalPoint}}
\begin{quote}

Either the p-value for the uncensored case or for both cases.
\end{quote}

\end{description}\end{quote}

\end{fulllineitems}

\index{getNoiseThreshold() (UnivariateLinearModelAnalysis method)}

\begin{fulllineitems}
\phantomsection\label{_generated/otpod.UnivariateLinearModelAnalysis:otpod.UnivariateLinearModelAnalysis.getNoiseThreshold}\pysiglinewithargsret{\bfcode{getNoiseThreshold}}{}{}
Accessor to the noise threshold.
\begin{quote}\begin{description}
\item[{Returns}] \leavevmode
\textbf{noiseThres} : float
\begin{quote}

The noise threhold if it exists, if not it returns \emph{None}.
\end{quote}

\end{description}\end{quote}

\end{fulllineitems}

\index{getOutputSample() (UnivariateLinearModelAnalysis method)}

\begin{fulllineitems}
\phantomsection\label{_generated/otpod.UnivariateLinearModelAnalysis:otpod.UnivariateLinearModelAnalysis.getOutputSample}\pysiglinewithargsret{\bfcode{getOutputSample}}{}{}
Accessor to the output sample.
\begin{quote}\begin{description}
\item[{Returns}] \leavevmode
\textbf{signals} : \href{http://doc.openturns.org/openturns-latest/sphinx/user\_manual/\_generated/openturns.NumericalSample.html\#openturns.NumericalSample}{\code{openturns.NumericalSample}}
\begin{quote}

The input sample which is the signal values.
\end{quote}

\end{description}\end{quote}

\end{fulllineitems}

\index{getR2() (UnivariateLinearModelAnalysis method)}

\begin{fulllineitems}
\phantomsection\label{_generated/otpod.UnivariateLinearModelAnalysis:otpod.UnivariateLinearModelAnalysis.getR2}\pysiglinewithargsret{\bfcode{getR2}}{}{}
Accessor to the R2 value.
\begin{quote}\begin{description}
\item[{Returns}] \leavevmode
\textbf{R2} : \href{http://doc.openturns.org/openturns-latest/sphinx/user\_manual/\_generated/openturns.NumericalPoint.html\#openturns.NumericalPoint}{\code{openturns.NumericalPoint}}
\begin{quote}

Either the R2 for the uncensored case or for both cases.
\end{quote}

\end{description}\end{quote}

\end{fulllineitems}

\index{getResiduals() (UnivariateLinearModelAnalysis method)}

\begin{fulllineitems}
\phantomsection\label{_generated/otpod.UnivariateLinearModelAnalysis:otpod.UnivariateLinearModelAnalysis.getResiduals}\pysiglinewithargsret{\bfcode{getResiduals}}{}{}
Accessor to the residuals.
\begin{quote}\begin{description}
\item[{Returns}] \leavevmode
\textbf{residuals} : \href{http://doc.openturns.org/openturns-latest/sphinx/user\_manual/\_generated/openturns.NumericalSample.html\#openturns.NumericalSample}{\code{openturns.NumericalSample}}
\begin{quote}

The residuals computed from the uncensored and censored linear
regression model. The first column corresponds with the uncensored case.
\end{quote}

\end{description}\end{quote}

\end{fulllineitems}

\index{getResidualsDistribution() (UnivariateLinearModelAnalysis method)}

\begin{fulllineitems}
\phantomsection\label{_generated/otpod.UnivariateLinearModelAnalysis:otpod.UnivariateLinearModelAnalysis.getResidualsDistribution}\pysiglinewithargsret{\bfcode{getResidualsDistribution}}{}{}
Accessor to the residuals distribution.
\begin{quote}\begin{description}
\item[{Returns}] \leavevmode
\textbf{distribution} : list of \href{http://doc.openturns.org/openturns-latest/sphinx/user\_manual/\_generated/openturns.Distribution.html\#openturns.Distribution}{\code{openturns.Distribution}}
\begin{quote}

The fitted distribution on the residuals, computed in the uncensored
and censored (if so) case.
\end{quote}

\end{description}\end{quote}

\end{fulllineitems}

\index{getSaturationThreshold() (UnivariateLinearModelAnalysis method)}

\begin{fulllineitems}
\phantomsection\label{_generated/otpod.UnivariateLinearModelAnalysis:otpod.UnivariateLinearModelAnalysis.getSaturationThreshold}\pysiglinewithargsret{\bfcode{getSaturationThreshold}}{}{}
Accessor to the saturation threshold.
\begin{quote}\begin{description}
\item[{Returns}] \leavevmode
\textbf{saturationThres} : float
\begin{quote}

The saturation threhold if it exists, if not it returns \emph{None}.
\end{quote}

\end{description}\end{quote}

\end{fulllineitems}

\index{getSlope() (UnivariateLinearModelAnalysis method)}

\begin{fulllineitems}
\phantomsection\label{_generated/otpod.UnivariateLinearModelAnalysis:otpod.UnivariateLinearModelAnalysis.getSlope}\pysiglinewithargsret{\bfcode{getSlope}}{}{}
Accessor to the slope of the linear regression model.
\begin{quote}\begin{description}
\item[{Returns}] \leavevmode
\textbf{slope} : \href{http://doc.openturns.org/openturns-latest/sphinx/user\_manual/\_generated/openturns.NumericalPoint.html\#openturns.NumericalPoint}{\code{openturns.NumericalPoint}}
\begin{quote}

The slope parameter for the uncensored and censored (if so) linear
regression model.
\end{quote}

\end{description}\end{quote}

\end{fulllineitems}

\index{getStandardError() (UnivariateLinearModelAnalysis method)}

\begin{fulllineitems}
\phantomsection\label{_generated/otpod.UnivariateLinearModelAnalysis:otpod.UnivariateLinearModelAnalysis.getStandardError}\pysiglinewithargsret{\bfcode{getStandardError}}{}{}
Accessor to the standard error of the estimate.
\begin{quote}\begin{description}
\item[{Returns}] \leavevmode
\textbf{stderr} : \href{http://doc.openturns.org/openturns-latest/sphinx/user\_manual/\_generated/openturns.NumericalPoint.html\#openturns.NumericalPoint}{\code{openturns.NumericalPoint}}
\begin{quote}

The standard error of the estimate for the uncensored and censored
(if so) linear regression model.
\end{quote}

\end{description}\end{quote}

\end{fulllineitems}

\index{getZeroMeanPValue() (UnivariateLinearModelAnalysis method)}

\begin{fulllineitems}
\phantomsection\label{_generated/otpod.UnivariateLinearModelAnalysis:otpod.UnivariateLinearModelAnalysis.getZeroMeanPValue}\pysiglinewithargsret{\bfcode{getZeroMeanPValue}}{}{}
Accessor to the Zero Mean test p-value.
\begin{quote}\begin{description}
\item[{Returns}] \leavevmode
\textbf{pValue} : \href{http://doc.openturns.org/openturns-latest/sphinx/user\_manual/\_generated/openturns.NumericalPoint.html\#openturns.NumericalPoint}{\code{openturns.NumericalPoint}}
\begin{quote}

Either the p-value for the uncensored case or for both cases.
\end{quote}

\end{description}\end{quote}

\end{fulllineitems}

\index{printResults() (UnivariateLinearModelAnalysis method)}

\begin{fulllineitems}
\phantomsection\label{_generated/otpod.UnivariateLinearModelAnalysis:otpod.UnivariateLinearModelAnalysis.printResults}\pysiglinewithargsret{\bfcode{printResults}}{}{}
Print results of the linear analysis in the terminal.

\end{fulllineitems}

\index{saveResults() (UnivariateLinearModelAnalysis method)}

\begin{fulllineitems}
\phantomsection\label{_generated/otpod.UnivariateLinearModelAnalysis:otpod.UnivariateLinearModelAnalysis.saveResults}\pysiglinewithargsret{\bfcode{saveResults}}{\emph{name}}{}
Save all analysis test results in a file.
\begin{quote}\begin{description}
\item[{Parameters}] \leavevmode
\textbf{name} : string
\begin{quote}

Name of the file or full path name.
\end{quote}

\end{description}\end{quote}
\paragraph{Notes}

The file can be saved as a csv file. Separations are made with tabulations.

If \emph{name} is the file name, then it is saved in the current working
directory.

\end{fulllineitems}


\end{fulllineitems}



\subsection{POD model}
\label{user_manual:pod-model}
\begin{longtable}{ll}
\hline
\endfirsthead

\multicolumn{2}{c}%
{{\textsf{\tablename\ \thetable{} -- continued from previous page}}} \\
\hline
\endhead

\hline \multicolumn{2}{|r|}{{\textsf{Continued on next page}}} \\ \hline
\endfoot

\endlastfoot


{\hyperref[_generated/otpod.UnivariateLinearModelPOD:otpod.UnivariateLinearModelPOD]{\emph{\code{UnivariateLinearModelPOD}}}}
 & 
Linear regression based POD.
\\
\hline\end{longtable}



\subsubsection{UnivariateLinearModelPOD}
\label{_generated/otpod.UnivariateLinearModelPOD:univariatelinearmodelpod}\label{_generated/otpod.UnivariateLinearModelPOD::doc}\index{UnivariateLinearModelPOD (class in otpod)}

\begin{fulllineitems}
\phantomsection\label{_generated/otpod.UnivariateLinearModelPOD:otpod.UnivariateLinearModelPOD}\pysiglinewithargsret{\strong{class }\bfcode{UnivariateLinearModelPOD}}{\emph{*args}}{}
Linear regression based POD.

\textbf{Available constructors:}

UnivariateLinearModelPOD(\emph{analysis=analysis, detection=detection})

UnivariateLinearModelPOD(\emph{inputSample, outputSample, detection, noiseThres,
saturationThres, resDistFact, boxCox})
\begin{quote}\begin{description}
\item[{Parameters}] \leavevmode
\textbf{analysis} : {\hyperref[_generated/otpod.UnivariateLinearModelAnalysis:otpod.UnivariateLinearModelAnalysis]{\emph{\code{UnivariateLinearModelAnalysis}}}}
\begin{quote}

Linear analysis object.
\end{quote}

\textbf{inputSample} : 2-d sequence of float
\begin{quote}

Vector of the defect sizes, of dimension 1.
\end{quote}

\textbf{outputSample} : 2-d sequence of float
\begin{quote}

Vector of the signals, of dimension 1.
\end{quote}

\textbf{detection} : float
\begin{quote}

Detection value of the signal.
\end{quote}

\textbf{noiseThres} : float
\begin{quote}

Value for low censored data. Default is None.
\end{quote}

\textbf{saturationThres} : float
\begin{quote}

Value for high censored data. Default is None
\end{quote}

\textbf{resDistFact} : \href{http://doc.openturns.org/openturns-latest/sphinx/user\_manual/\_generated/openturns.DistributionFactory.html\#openturns.DistributionFactory}{\code{openturns.DistributionFactory}}
\begin{quote}

Distribution hypothesis followed by the residuals. Default is None.
\end{quote}

\textbf{boxCox} : bool or float
\begin{quote}

Enable or not the Box Cox transformation. If boxCox is a float, the Box
Cox transformation is enabled with the given value. Default is False.
\end{quote}

\end{description}\end{quote}
\paragraph{Notes}

This class aims at building the POD based on a linear regression
model. If a linear analysis has been launched, it can be used as prescribed 
in the first constructor. It can be noticed that, in this case, with the
default parameters of the linear analysis, the POD will corresponds with the
linear regression model associated to a Gaussian hypothesis on the residuals.

Otherwise, all parameters can be given as in the second constructor.

Following the given distribution in \emph{resDistFact}, the POD model is built
different hypothesis:
\begin{itemize}
\item {} 
if \emph{resDistFact = None}, it corresponds with Berens-Binomial. This
is the default case.

\item {} 
if \emph{resDistFact} = \href{http://doc.openturns.org/openturns-latest/sphinx/user\_manual/\_generated/openturns.NormalFactory.html\#openturns.NormalFactory}{\code{openturns.NormalFactory}}, it corresponds with Berens-Gauss.

\item {} 
if \emph{resDistFact} = \{\href{http://doc.openturns.org/openturns-latest/sphinx/user\_manual/\_generated/openturns.KernelSmoothing.html\#openturns.KernelSmoothing}{\code{openturns.KernelSmoothing}},
\href{http://doc.openturns.org/openturns-latest/sphinx/user\_manual/\_generated/openturns.WeibullFactory.html\#openturns.WeibullFactory}{\code{openturns.WeibullFactory}}, ...\}, the confidence interval is
built by bootstrap.

\end{itemize}
\paragraph{Methods}

\begin{longtable}{ll}
\hline
\endfirsthead

\multicolumn{2}{c}%
{{\textsf{\tablename\ \thetable{} -- continued from previous page}}} \\
\hline
\endhead

\hline \multicolumn{2}{|r|}{{\textsf{Continued on next page}}} \\ \hline
\endfoot

\endlastfoot


{\hyperref[_generated/otpod.UnivariateLinearModelPOD:otpod.UnivariateLinearModelPOD.computeDetectionSize]{\emph{\code{computeDetectionSize}}}}(*args, **kwargs)
 & 
Compute the detection size for a given probability level.
\\
\hline
{\hyperref[_generated/otpod.UnivariateLinearModelPOD:otpod.UnivariateLinearModelPOD.drawPOD]{\emph{\code{drawPOD}}}}(*args, **kwargs)
 & 
Draw the POD curve.
\\
\hline
{\hyperref[_generated/otpod.UnivariateLinearModelPOD:otpod.UnivariateLinearModelPOD.getPODCLModel]{\emph{\code{getPODCLModel}}}}({[}confidenceLevel{]})
 & 
Accessor to the POD model at a given confidence level.
\\
\hline
{\hyperref[_generated/otpod.UnivariateLinearModelPOD:otpod.UnivariateLinearModelPOD.getPODModel]{\emph{\code{getPODModel}}}}()
 & 
Accessor to the POD model.
\\
\hline
{\hyperref[_generated/otpod.UnivariateLinearModelPOD:otpod.UnivariateLinearModelPOD.getSimulationSize]{\emph{\code{getSimulationSize}}}}()
 & 
Accessor to the simulation size.
\\
\hline
{\hyperref[_generated/otpod.UnivariateLinearModelPOD:otpod.UnivariateLinearModelPOD.run]{\emph{\code{run}}}}()
 & 
Build the POD models.
\\
\hline
{\hyperref[_generated/otpod.UnivariateLinearModelPOD:otpod.UnivariateLinearModelPOD.setSimulationSize]{\emph{\code{setSimulationSize}}}}(size)
 & 
Accessor to the simulation size
\\
\hline\end{longtable}

\index{computeDetectionSize() (UnivariateLinearModelPOD method)}

\begin{fulllineitems}
\phantomsection\label{_generated/otpod.UnivariateLinearModelPOD:otpod.UnivariateLinearModelPOD.computeDetectionSize}\pysiglinewithargsret{\bfcode{computeDetectionSize}}{\emph{*args}, \emph{**kwargs}}{}
Compute the detection size for a given probability level.
\begin{quote}\begin{description}
\item[{Parameters}] \leavevmode
\textbf{probabilityLevel} : float
\begin{quote}

The probability level for which the defect size is computed.
\end{quote}

\textbf{confidenceLevel} : float
\begin{quote}

The confidence level associated to the given probability level the
defect size is computed. Default is None.
\end{quote}

\item[{Returns}] \leavevmode
\textbf{result} : collection of \href{http://doc.openturns.org/openturns-latest/sphinx/user\_manual/\_generated/openturns.NumericalPointWithDescription.html\#openturns.NumericalPointWithDescription}{\code{openturns.NumericalPointWithDescription}}
\begin{quote}

A list of NumericalPointWithDescription containing the detection size
computing for each case.
\end{quote}

\end{description}\end{quote}

\end{fulllineitems}

\index{drawPOD() (UnivariateLinearModelPOD method)}

\begin{fulllineitems}
\phantomsection\label{_generated/otpod.UnivariateLinearModelPOD:otpod.UnivariateLinearModelPOD.drawPOD}\pysiglinewithargsret{\bfcode{drawPOD}}{\emph{*args}, \emph{**kwargs}}{}
Draw the POD curve.
\begin{quote}\begin{description}
\item[{Parameters}] \leavevmode
\textbf{probabilityLevel} : float
\begin{quote}

The probability level for which the defect size is computed. Default
is None.
\end{quote}

\textbf{confidenceLevel} : float
\begin{quote}

The confidence level associated to the given probability level the
defect size is computed. Default is None.
\end{quote}

\textbf{defectMin, defectMax} : float
\begin{quote}

Define the interval where the curve is plotted. Default : min and
max values of the inputSample.
\end{quote}

\textbf{nbPt} : int
\begin{quote}

The number of points to draw the curves. Default is 100.
\end{quote}

\textbf{name} : string
\begin{quote}

name of the figure to be saved with \emph{transparent} option sets to True
and \emph{bbox\_inches='tight'}. It can be only the file name or the 
full path name. Default is None.
\end{quote}

\item[{Returns}] \leavevmode
\textbf{fig} : \href{http://matplotlib.org/api/figure\_api.html}{matplotlib.figure}
\begin{quote}

Matplotlib figure object.
\end{quote}

\textbf{ax} : \href{http://matplotlib.org/api/axes\_api.html}{matplotlib.axes}
\begin{quote}

Matplotlib axes object.
\end{quote}

\end{description}\end{quote}

\end{fulllineitems}

\index{getPODCLModel() (UnivariateLinearModelPOD method)}

\begin{fulllineitems}
\phantomsection\label{_generated/otpod.UnivariateLinearModelPOD:otpod.UnivariateLinearModelPOD.getPODCLModel}\pysiglinewithargsret{\bfcode{getPODCLModel}}{\emph{confidenceLevel=0.95}}{}
Accessor to the POD model at a given confidence level.
\begin{description}
\item[{confidenceLevel}] \leavevmode{[}float{]}
The confidence level the POD must be computed. Default is 0.95

\end{description}
\begin{quote}\begin{description}
\item[{Returns}] \leavevmode
\textbf{PODModelCl} : \href{http://doc.openturns.org/openturns-latest/sphinx/user\_manual/\_generated/openturns.NumericalMathFunction.html\#openturns.NumericalMathFunction}{\code{openturns.NumericalMathFunction}}
\begin{quote}

The function which computes the probability of detection for a given
defect value at the confidence level given as parameter.
\end{quote}

\end{description}\end{quote}

\end{fulllineitems}

\index{getPODModel() (UnivariateLinearModelPOD method)}

\begin{fulllineitems}
\phantomsection\label{_generated/otpod.UnivariateLinearModelPOD:otpod.UnivariateLinearModelPOD.getPODModel}\pysiglinewithargsret{\bfcode{getPODModel}}{}{}
Accessor to the POD model.
\begin{quote}\begin{description}
\item[{Returns}] \leavevmode
\textbf{PODModel} : \href{http://doc.openturns.org/openturns-latest/sphinx/user\_manual/\_generated/openturns.NumericalMathFunction.html\#openturns.NumericalMathFunction}{\code{openturns.NumericalMathFunction}}
\begin{quote}

The function which computes the probability of detection for a given
defect value.
\end{quote}

\end{description}\end{quote}

\end{fulllineitems}

\index{getSimulationSize() (UnivariateLinearModelPOD method)}

\begin{fulllineitems}
\phantomsection\label{_generated/otpod.UnivariateLinearModelPOD:otpod.UnivariateLinearModelPOD.getSimulationSize}\pysiglinewithargsret{\bfcode{getSimulationSize}}{}{}
Accessor to the simulation size.

\end{fulllineitems}

\index{run() (UnivariateLinearModelPOD method)}

\begin{fulllineitems}
\phantomsection\label{_generated/otpod.UnivariateLinearModelPOD:otpod.UnivariateLinearModelPOD.run}\pysiglinewithargsret{\bfcode{run}}{}{}
Build the POD models.
\paragraph{Notes}

This method build the linear model for the uncensored or censored case
depending of the input parameters. Then it builds the POD model
following the given residuals distribution factory.

\end{fulllineitems}

\index{setSimulationSize() (UnivariateLinearModelPOD method)}

\begin{fulllineitems}
\phantomsection\label{_generated/otpod.UnivariateLinearModelPOD:otpod.UnivariateLinearModelPOD.setSimulationSize}\pysiglinewithargsret{\bfcode{setSimulationSize}}{\emph{size}}{}
Accessor to the simulation size
\begin{quote}\begin{description}
\item[{Parameters}] \leavevmode
\textbf{size} : int
\begin{quote}

The size of the simulation used to compute the confidence interval.
\end{quote}

\end{description}\end{quote}

\end{fulllineitems}


\end{fulllineitems}



\subsection{Tools}
\label{user_manual:tools}
\begin{longtable}{ll}
\hline
\endfirsthead

\multicolumn{2}{c}%
{{\textsf{\tablename\ \thetable{} -- continued from previous page}}} \\
\hline
\endhead

\hline \multicolumn{2}{|r|}{{\textsf{Continued on next page}}} \\ \hline
\endfoot

\endlastfoot


{\hyperref[_generated/otpod.DataHandling:otpod.DataHandling]{\emph{\code{DataHandling}}}}
 & 
Static methods for data handling.
\\
\hline\end{longtable}



\subsubsection{DataHandling}
\label{_generated/otpod.DataHandling::doc}\label{_generated/otpod.DataHandling:datahandling}\index{DataHandling (class in otpod)}

\begin{fulllineitems}
\phantomsection\label{_generated/otpod.DataHandling:otpod.DataHandling}\pysigline{\strong{class }\bfcode{DataHandling}}
Static methods for data handling.
\paragraph{Methods}

\begin{longtable}{ll}
\hline
\endfirsthead

\multicolumn{2}{c}%
{{\textsf{\tablename\ \thetable{} -- continued from previous page}}} \\
\hline
\endhead

\hline \multicolumn{2}{|r|}{{\textsf{Continued on next page}}} \\ \hline
\endfoot

\endlastfoot


{\hyperref[_generated/otpod.DataHandling:otpod.DataHandling.filterCensoredData]{\emph{\code{filterCensoredData}}}}(defects, signals, ...)
 & 
Sort defect sizes with respect to the censored signals.
\\
\hline\end{longtable}

\index{filterCensoredData() (DataHandling static method)}

\begin{fulllineitems}
\phantomsection\label{_generated/otpod.DataHandling:otpod.DataHandling.filterCensoredData}\pysiglinewithargsret{\strong{static }\bfcode{filterCensoredData}}{\emph{defects}, \emph{signals}, \emph{noiseThres}, \emph{saturationThres}}{}
Sort defect sizes with respect to the censored signals.
\begin{quote}\begin{description}
\item[{Parameters}] \leavevmode
\textbf{defects} : 2-d sequence of float
\begin{quote}

Vector of the defect sizes.
\end{quote}

\textbf{signals} : 2-d sequence of float
\begin{quote}

Vector of the signals, of dimension 1.
\end{quote}

\textbf{noiseThres} : float
\begin{quote}

Value for low censored data. Default is None.
\end{quote}

\textbf{saturationThres} : float
\begin{quote}

Value for high censored data. Default is None
\end{quote}

\item[{Returns}] \leavevmode
\textbf{defectsUnc} : 2-d sequence of float
\begin{quote}

Vector of the defect sizes in the uncensored area.
\end{quote}

\textbf{defectsNoise} : 2-d sequence of float
\begin{quote}

Vector of the defect sizes in the noisy area.
\end{quote}

\textbf{defectsSat} : 2-d sequence of float
\begin{quote}

Vector of the defect sizes in the saturation area.
\end{quote}

\textbf{signalsUnc} : 2-d sequence of float
\begin{quote}

Vector of the signals in the uncensored area.
\end{quote}

\end{description}\end{quote}
\paragraph{Notes}

The data are sorted in three different vectors whether they belong to
the noisy area, the uncensored area or the saturation area.

\end{fulllineitems}


\end{fulllineitems}



\section{Examples}
\label{examples::doc}\label{examples:examples}

\chapter{Indices and tables}
\label{index:indices-and-tables}\begin{itemize}
\item {} 
\DUspan{xref,std,std-ref}{genindex}

\item {} 
\DUspan{xref,std,std-ref}{modindex}

\item {} 
\DUspan{xref,std,std-ref}{search}

\end{itemize}



\renewcommand{\indexname}{Index}
\printindex
\end{document}
